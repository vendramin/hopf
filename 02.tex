\subsection{Modules}

We now give the usual definition of modules over
algebras but using commutative diagrams.

\begin{definition}
    \index{Module}
    Let $A$ be an algebra. A \emph{left $A$-module} is a vector 
    space $M$ together with a linear map 
    $\gamma\colon A\otimes M\to M$ such that
    the diagrams 
    \[
    \begin{tikzcd}
	A\otimes A\otimes M & {A\otimes M} \\
	{A\otimes M} & {M}
	\arrow["m\otimes\id", from=1-1, to=1-2]
	\arrow["\id\otimes\gamma"', from=1-1, to=2-1]
	\arrow["\gamma", from=1-2, to=2-2]
	\arrow["\gamma"', from=2-1, to=2-2]
    \end{tikzcd}
    \quad\text{and}\quad 
    \begin{tikzcd}
	K\otimes M & {A\otimes M} \\
	&{M}
	\arrow["u\otimes\id", from=1-1, to=1-2]
	\arrow["\gamma", from=1-2, to=2-2]
	\arrow["\simeq"', from=1-1, to=2-2]
    \end{tikzcd}
    \]
    both commute.
\end{definition}

\begin{definition}
    \index{Module!homomorphism}
    Let $M$ and $N$ be left $A$-modules. 
    A linear map $f\colon M\to N$ is a left module \emph{homomorphism}
    if $f(a\cdot m)=a\cdot f(m)$ for all $a\in A$ and $m\in M$. 
\end{definition}

The category of left $A$-modules is denoted $\lmod{A}$. 

Right modules and right module homomorphisms are defined similarly. We leave it to the reader to write the definitions explicitly. The category of right $A$-modules is denoted $\rmod{A}$. 


\subsection{Comodules}

A comodule over a coalgebra is the dual 
concept of a module over an algebra. Instead of an algebra acting on a vector space, a coalgebra coacts on it. 

\begin{definition}
\label{def:comodule}
\index{Comodule}
    Let $C$ be a coalgebra. A \emph{left $C$-comodule} 
    is a vector space $M$ together with 
    a linear map $\delta\colon M\to C\otimes M$
    such that the diagrams 
    \[
    \begin{tikzcd}
	M & {C\otimes M} \\
	{C\otimes M} & {C\otimes C\otimes M}
	\arrow["\delta", from=1-1, to=1-2]
	\arrow["\delta"', from=1-1, to=2-1]
	\arrow["\Delta\otimes\id", from=1-2, to=2-2]
	\arrow["\id\otimes\delta"', from=2-1, to=2-2]
    \end{tikzcd}
    \quad\text{and}\quad 
    \begin{tikzcd}
	M & {C\otimes M} \\
	&{K\otimes M}
	\arrow["\delta", from=1-1, to=1-2]
	\arrow["\epsilon\otimes\id", from=1-2, to=2-2]
	\arrow["\simeq"', from=1-1, to=2-2]
    \end{tikzcd}
    \]
    commute. In Sweedler's notation, 
    \begin{align*}
        \delta(m)&=\sum m_{-1}\otimes m_0\in C\otimes M.
%        (\delta\otimes\id)\delta(m)&=m_{-2}\otimes m_{-1}\otimes m_0\in C\otimes C\otimes M.
    \end{align*}
    We write  
    \begin{align*}
        \sum m_{-2}\otimes m_{-1}\otimes m_{0}&
        =\sum m_{-1}\otimes m_{0,-1}\otimes m_{0,0}
        =\sum m_{-1,1}\otimes m_{-1,2}\otimes m_0
    \end{align*}
    for all $m\in M$. Note that 
    the second diagram says that $\sum\epsilon(m_{-1})m_0=m$ for all 
    $m\in M$. 
\end{definition}

Similarly, a \emph{right $C$-comodule} is defined
by a linear map $\delta\colon M\to M\otimes C$ such that
the diagrams 
    \[
    \begin{tikzcd}
	M & {M\otimes C} \\
	{M\otimes C} & {M\otimes C\otimes C}
	\arrow["\delta", from=1-1, to=1-2]
	\arrow["\delta"', from=1-1, to=2-1]
	\arrow["\id\otimes\Delta", from=1-2, to=2-2]
	\arrow["\delta\otimes\id"', from=2-1, to=2-2]
    \end{tikzcd}
    \quad\text{and}\quad 
    \begin{tikzcd}
	M & {M\otimes K} \\
	&{M\otimes C}
	\arrow["\delta", from=1-1, to=1-2]
	\arrow["\id\otimes\epsilon", from=1-2, to=2-2]
	\arrow["\simeq"', from=1-1, to=2-2]
\end{tikzcd}
    \]
commute. In Sweedler's notation, 
\[
\delta(m)=\sum m_0\otimes m_1\in M\otimes C.
\]
We write 
    \begin{align*}
        \sum m_{-1}\otimes m_0\otimes m_1&=
        \sum m_0\otimes m_{1,1}\otimes m_{1,2}
        =\sum m_{0,0}\otimes m_{0,1}\otimes m_1
    \end{align*}
    for all $m\in M$. The second diagram says that 
    $\sum m_0\epsilon(m_1)=m$ for all $m\in M$. 
    
Note that Sweedler's notation for left and right comodules is compatible with the notation we introduced before for coalgebras. This is fantastic!

\begin{example}
    Any coalgebra is a comodule with coaction $\Delta$. 
\end{example}

%How do comodules over group algebras look like?  

\begin{example}
    Let $G$ be a finite group and $C=K[G]$ with 
    the coalgebra structure given by $\Delta(g)=g\otimes g$ and 
    $\epsilon(g)=1$ for all $g\in G$. Let $M$ be a left $C$-comodule
\end{example}
    
\begin{exercise}
\label{xca:sum_comodules}
Prove that the sum of comodules is a comodule.     
\end{exercise}

\begin{exercise}
    \label{xca:1.6.4a}
    Let $C$ be a coalgebra. Prove that 
    if $M$ is a right $C$-comodule, then 
    $M$ is a left $C^*$-module with
    \[
    f\cdot m=\sum f(m_1)m_0.
    \]
\end{exercise}

\begin{exercise}
    \label{xca:1.6.4b}
    Let $A$ be a finite-dimensional algebra. Prove that 
    if $M$ is a left $A$-module, then 
    $M$ is a right $A^*$-comodule. 
\end{exercise}


\begin{definition}
  \index{Comodule!homomorphism}
  Let $C$ be a coalgebra and $N$ and $N_1$ be left $C$-comodules with
  coactions $\delta$ and $\delta_1$, respectively. 
  A linear map $f\colon N\to N_1$ is a comodule \emph{homomorphism}
  if the diagram 
  \[
  \begin{tikzcd}
	N & {C\otimes N} \\
	N_1 & {C\otimes N_1}
	\arrow["\delta", from=1-1, to=1-2]
	\arrow["\delta_1"', from=2-1, to=2-2]
	\arrow["\id\otimes f", from=1-2, to=2-2]
	\arrow["f"', from=1-1, to=2-1]
  \end{tikzcd}
  \]
  commutes, that is $(\id\otimes f)\delta=\delta_1 f$. 
\end{definition}

\begin{definition}
  \index{Subcomodule}
  Let $C$ be a coalgebra and $N$ be a left $C$-comodule. A subspace
  $M$ of $N$ is a \emph{subcomodule} of $N$ if $\delta(M)\subseteq C\otimes M$. 
\end{definition}

% Let $m\in M$ and $\{m_1,\dots,m_k\}$ be a basis of $A\cdot m$. 
% Thus $a\cdot m=\sum_{i=1}^kf_i(a)m_i$
% then $\rho(m)=\sum_{i=1}^n m_i\otimes f_i$
\subsection{The fundamental theorem of coalgebras}

\begin{definition}
    \index{Subcoalgebra}
    Let $C$ be a coalgebra. A \emph{subcoalgebra} is
    a subspace $D$ of $C$ such that $\Delta(D)\subseteq D\otimes D$. 
\end{definition}

Note that the definition of a subcoalgebra does not involve the counit. Why?

The reader should verify that a subcoalgebra is indeed a coalgebra with the appropriately restricted maps. Moreover, the inclusion map of a subcoalgebra into a coalgebra is a coalgebra homomorphism.

\begin{theorem}[fundamental theorem of coalgebras]
\index{Fundamental theorem of coalgebras}
\label{thm:fundamental}
    Let $C$ be a coalgebra.
    \begin{enumerate}
        \item Let $M$ a right $C$-comodule and 
    $m\in M$. Then there exists a finite-dimensional
    subcomodule $N$ of $M$ such that $m\in N$. 
        \item Let $c\in C$. Then there exists a finite-dimensional subcoalgebra $D$ of
        $C$ such that $c\in D$. 
    \end{enumerate}
\end{theorem}

\begin{proof}\
    \begin{enumerate}
            \item Let $\{c_i:i\in I\}$ be a basis of $C$. Let $\rho\colon M\to M\otimes C$, $m\mapsto\sum w_i\otimes c_i$, where all but finitely many $w_i$ are
            zero. For each $i$, write 
            \[
            \Delta(c_i)=\sum\alpha_{ijk}c_j\otimes c_j
            \]
            where $\lambda_{ijk}\in K$. Then
            \begin{align*}
                \sum\rho(w_i)\otimes c_i
                =(\rho\otimes\id)\rho(m)
                =(\id\otimes\Delta)\rho(m)
                =\sum w_i\otimes\alpha_{ijk}c_j\otimes c_k.
            \end{align*}
            Comparing the coefficient of $c_k$, we obtain that
            \[
            \rho(w_k)=\sum w_i\otimes\alpha_{ijk}c_j.
            \]
            Let $N=\operatorname{span}_K\{m,w_i\}\subseteq M$. 
            Then $N$ is a subcomodule containing $m$. 
        \item Apply the first part to the $C$-comodule 
        $M=C$ with $\rho=\Delta$. Then there exists 
        a finite-dimensional subspace $V$ of $C$ containing $c$. 
        Moreover, $\Delta(V)\subseteq V\otimes C$. Let $\{v_1,\dots,v_n\}$ be a basis of $V$. For each $j$, write
        \[
        \Delta(v_j)=\sum v_i\otimes c_{ij}. 
        \]
        Then $\Delta(c_{ij})=\sum_kc_{ik}\otimes c_{kj}$. Let 
        $D=\operatorname{span}_K\{v_1,\dots,v_n,c_{ij}\}$. Then the claim follows since 
        $\Delta(D)\subseteq D\otimes D$ and $V\subseteq D$.\qedhere 
    \end{enumerate}
\end{proof}

Theorem~\ref{thm:fundamental} immediately implies that every simple coalgebra is finite-dimensional. Similarly, every simple comodule is finite-dimensional.

\begin{exercise}
    Prove that Theorem~\ref{thm:fundamental} extends to arbitrary finite sums of elements.
\end{exercise}

% \begin{theorem}
% \index{Fundamental theorem of coalgebras}
%     Let $C$ be a coalgebra and $c\in C$. Then $c$ is contained
%     in a finite-dimensional subcoalgebra of $C$. 
% \end{theorem}

% Cartier 2.6.1

\section{Bialgebras and Hopf algebras}

\begin{definition}
    \index{Bialgebra}
    A \emph{bialgebra} is a vector space $B$ that is both an algebra and a coalgebra such that the comultiplication and the counit are algebra homomorphisms. 
\end{definition}

\begin{exercise}
\label{xca:bialgebra}
    Let $B$ be both an algebra and a coalgebra. Prove that 
    $B$ is a bialgebra if and only if 
    multiplication and the unit are coalgebra homomorphisms. 
\end{exercise}

\begin{definition}
\index{Bialgebra!homomorphism}
A bialgebra \emph{homomorphism} is a map that is both an algebra and
a coalgebra homomorphism. 
\end{definition} 

\begin{definition}
    \index{Hopf algebra}
    \index{Antipode}
    A \emph{Hopf algebra} is a bialgebra $H$ 
    together with a linear map 
    \[
    S\colon H\to H,
    \]
    called the \emph{antipode}, such that 
    the diagram 
\begin{equation}
\label{eq:antipode}
\begin{tikzcd}
	& {H\otimes H} && {H\otimes H} \\
	H && K && H \\
	& {H\otimes H} && {H\otimes H}
	\arrow["{S\otimes\id}", from=1-2, to=1-4]
	\arrow["m", from=1-4, to=2-5]
	\arrow["\Delta", from=2-1, to=1-2]
	\arrow["\epsilon", from=2-1, to=2-3]
	\arrow["\Delta"', from=2-1, to=3-2]
	\arrow["u", from=2-3, to=2-5]
	\arrow["{\id\otimes S}", from=3-2, to=3-4]
	\arrow["m"', from=3-4, to=2-5]
\end{tikzcd}
\end{equation}
    commutes. 
\end{definition}

Using Sweedler’s notation, the commutativity of the diagram~\eqref{eq:antipode} can be expressed as
\[
S(h_{1})h_{2}=h_{1}S(h_{2})=\epsilon(h)1
\]
for all $h\in H$. 

A routine argument shows the uniqueness of the antipode, when it exists.

\begin{exercise}
    \label{xca:convolution}
    Let $A$ be an algebra and $C$ a coalgebra. 
    Prove that the set 
    of linear homomorphism $\Hom(C,A)$ together with
    the \emph{convolution product}
    \[
    (f*g)(x) = \sum f(x_1)g(x_2)
    \]
    is an algebra. 
\end{exercise}

\begin{exercise}
    \label{xca:antipode}
    Let $H$ be a Hopf algebra. Prove that
    the antipode is the inverse of 
    $\id$ with respect to the convolution 
    product of $\Hom(H,H)$ of Exercise~\ref{xca:convolution}.
\end{exercise}

\begin{definition}
\index{Hopf algebra!homomorphism}
A Hopf algebra \emph{homomorphism} 
is a bialgebra homomorphism that commutes with the antipode. 
\end{definition}
% Sweedler 4.0.1

\begin{theorem}
\label{thm:antipode}
    Let $H$ be a Hopf algebra. Then
    \begin{align*}
        S(xy)=S(y)S(x), && S(1)=1, &&
        \Delta(S(x))=S(x_{2})\otimes S(x_{1}),
        && \epsilon(S(x))=\epsilon(x)
    \end{align*}
    for all $x\in H$. 
\end{theorem}

\begin{exercise}
    Prove Theorem~\ref{thm:antipode}.
\end{exercise}

Theorem~\ref{thm:antipode} states that the antipode is both an anti-algebra and
an anti-coalgebra homomorphism.

\begin{exercise}
    Let $H$ be a commutative or cocommutative Hopf algebra. 
    Prove that $S^2=\id$. 
\end{exercise}

% cartier 3.1.1

%https://en.wikipedia.org/wiki/Hopf_algebra

\begin{example}[group algebra]
\index{Group algebra}
    Let $G$ be a finite group. Then $\C[G]$ with 
    $\Delta(g) = g\otimes g$,  
    $\epsilon(g)=1$ and $S(g)=g^{-1}$ for all $g\in G$ 
    is a Hopf algebra. Note that $\C[G]$ is cocommutative and 
    $\C[G]$ is commutative if and only if $G$ is abelian. 
\end{example}

% \begin{example}
%     Let $G$ be a finite group. Then the set 
%     $\C^G$ of maps $G\to\C$ is a Hopf algebra
%     with point-wise addition and multiplication and 
%     $\Delta(f)(x,y)=f(xy)$, $\epsilon(f)=f(1)$ and 
%     $S(f)(x)=f(x^{-1})$ 
% \end{example}

\begin{example}[trigonometric Hopf algebra]
    % Let $A$ be a commutative algebra. Then $C(A)=\{(x,y):x^2+y^2=1\}$ is a group with 
    % \[
    % (a,b)(c,d)=(ac-bd,ad+bc).
    % \]
    The algebra $H$ with generators $s,c$ and 
    relations $c^2+s^2-1$ is a Hopf algebra with 
    \begin{align*}
    \Delta(c)=c\otimes c-s\otimes s,
    && 
    \Delta(s)=c\otimes s+s\otimes c,
    &&
    \epsilon(c)=1,
    &&
    \epsilon(s)=0,
    &&
    S(c)=c,
    &&
    S(s)=-s.    
    \end{align*}
\end{example}

\begin{example}
\index{Tensor algebra} 
Let $V$ be a $K$-vector space. Let $T^0(V)=K$ and
\[
  T^n(V)=V^{\otimes n}=V\otimes\cdots\otimes V\text{ ($n$-times) }
\]
for $n\geq1$. For $n,m\geq0$, 
the isomorphisms \[
  T^n(V)\otimes T^m(V)\simeq T^{n+m}(V)
\]
induce an associative product on $T(V)=\oplus_{n\geq0}T^n(V)$, namely
\[
  (x_1\otimes\cdots\otimes x_n)(x_{n+1}\otimes\cdots\otimes x_{n+m})
  =x_1\otimes\cdots\otimes x_n\otimes x_{n+1}\otimes\cdots\otimes x_{n+m}
\]
for $x_1,\dots,x_n,x_{n+1},\dots,x_{n+m}\in V$. 

Let $\iota\colon V\to T(V)$ be the canonical embedding. Since 
$x_1\otimes\cdots\otimes x_n=\iota(x_1)\cdots\iota(x_n)$, we can identity 
$x_1\otimes\cdots\otimes x_n$ with $x_1\cdots x_n$. 

The vector space $T(V)$ with the product mentioned before is a 
graded algebra. It is called the \emph{tensor algebra} of $V$. 
\end{example}

\begin{exercise}
  Let $V$ be a vector space, $A$ an algebra and 
  $f\colon V\to A$ a linear map. Prove that there exists a unique
  algebra homomorphism $T(V)\to A$ such that the diagram
  \[\begin{tikzcd}
    V & A \\
    {T(V)}
    \arrow["f", from=1-1, to=1-2]
    \arrow["\iota"', from=1-1, to=2-1]
    \arrow[dashed, from=2-1, to=1-2]
  \end{tikzcd}
\]
  commutes. 
\end{exercise}

\begin{exercise}
  Let $V$ be a vector space with basis $\{v_i:i\in I\}$. Prove that $T(V)$ is
  isomorphic to the free algebra on $I$. 
\end{exercise}

% II.5.1, Kassel, page 34

\begin{example}
\index{Universal enveloping algebra}
    Let $\mathfrak{g}$ be a Lie algebra. The \emph{enveloping algebra} $H=\mathcal{U}(\mathfrak{g})$ is a Hopf algebra with
    \[
    \Delta(x)=x\otimes 1+1\otimes x,\quad 
    \epsilon(x)=0,\quad 
    S(x)=-x
    \]
    for all $x\in\mathfrak{g}$. 
\end{example}

\begin{exercise}
\label{xca:primitives}
    Let $H$ be a Hopf algebra. Let $x\in H$ be such that 
    $\Delta(x)=x\otimes 1+1\otimes x$. Prove that 
    $\epsilon(x)=0$ and $S(x)=-x$. 
\end{exercise}

% Apply 1\otimes\epsilon to \Delta(x). 
% Then use the definition of $S$. 

The following example is the smallest example of a Hopf algebra that is both non-commutative and non-cocommutative.

\begin{example}[Sweedler]
\index{Sweedler!Hopf algebra}
Assume that $\operatorname{char}K\ne2$. Let $H_4$ be
the algebra with generators $1,g,x,gx$ and relations
\[
g^2=1,\quad x^2=0,\quad xg=-gx.
\]
Then $H_4$ is a Hopf algebra with
\begin{align*}
&\Delta(g)=g\otimes g,
&&\Delta(x)=x\otimes 1+1\otimes x,
&&\epsilon(g)=1,
&&\epsilon(x)=0,
&&S(g)=g^{-1},
&&S(x)=-gx.    
\end{align*}
Note that $H_4$ is non-commutative and non-cocommutative. Moreover, 
$S$ has order four. 
\end{example}

\section{Actions and coactions on (co)algebras}

\begin{definition}
\label{def:module_algebra}
\index{Module!algebra}
Let $H$ be a Hopf algebra. A left \emph{$H$-module-algebra} is an algebra
$A$ with a left $H$-module structure such that
\[
h\rightarrow(ab)=\sum (h_{1}\rightarrow a)(h_{2}\rightarrow b)
\quad \text{and}\quad 
h\rightarrow1=\epsilon(h)1
\]
for all $h\in H$ and $a,b\in A$. 
\end{definition}

Similarly, one defines right $H$-module-algebras. It is an
algebra with a right $H$-module structure such that 
\[
(ab)\leftarrow
h=\sum (a\leftarrow h_{1})(b\leftarrow h_{2})
\quad\text{and}\quad 1\leftarrow h=\epsilon(h)1 
\]
for
all $h\in H$ and $a,b\in A$. 

\begin{exercise}
\label{xca:dualH_on_H}
Let $H$ be a Hopf algebra.  Prove that $H^*$ is a left $H$-module-algebra
with  $\langle h\rightharpoonup f,x\rangle=\langle f,xh\rangle$ for all $f\in
H^*$ and $h,x\in H$.  
\end{exercise}

Similarly, $H^*$ is a right $H$-module-algebra
with $\langle f\leftharpoonup h,x\rangle=\langle f,xh\rangle$.

\begin{exercise}
\label{xca:adjoint}
\index{Adjoint!action}
Let $H$ be a Hopf algebra. Prove that $H$ is a left $H$-module algebra with 
the \emph{left
adjoint action}, that is 
$a\rightarrow x=\sum a_{1}xS(a_{2})$
for $x,a\in H$.
\end{exercise}

Similarly, the \emph{right
adjoint action} of $H$ is given by 
$x\leftarrow a=\sum S(a_{1})xa_{2}$ for $x,a\in H$.  

\begin{exercise}
    How do the adjoint actions look for 
    group algebras $K[G]$ and enveloping algebras 
    $\mathcal{U}(\mathfrak{g})$ of Lie algebras? 
\end{exercise}

% \begin{example}
% Let $G$ be a finite group and $H=K[G]$. In this case, 
% the left adjoint action is $a\rightarrow x=axa^{-1}$.
% \end{example} 

% \begin{example}
% Let $\mathfrak{g}$ be a Lie algebra and $\mathcal{U}(\mathfrak{g})$. The left adjoint action
% is 
% $a\rightarrow x=ax-xa$. 
% \end{example}

The following exercise introduces the 
\emph{left smash product}. 

\begin{exercise}
\label{xca:left_smash}
\index{Smash!product}
Let $H$ be a bialgebra and $(A,\rightarrow)$ be a left $H$-module-algebra.
Prove that there exists an algebra structure on $A\otimes H$ given by
\[
(a\otimes h)(b\otimes g)=\sum a(h_{1}\rightarrow b)\otimes h_{2}g
\]
and unit $1\otimes1$. Moreover, 
the maps $A\to A\otimes H$, $a\mapsto a\otimes1$,
and $H\to A\otimes H$, $h\mapsto 1\otimes h$ 
are algebra embeddings.
\end{exercise}

If $A$ is a right $H$-module-algebra, then 
the \emph{right smash product} is the algebra structure 
on $H\otimes A$ given by 
\[
(h\otimes a)(g\otimes b)=\sum hg_{1}\otimes(a\leftarrow g_{2})b
\]
and unit $1\otimes1$.

% \begin{exercise}
% \index{smash product!right}
% Let $H$ be a Hopf algebra and $(A,\leftarrow)$ be a right $H$-module-algebra.
% Prove that there exists an algebra structure on $H\otimes A$ given by 
% \[
% (h\otimes a)(g\otimes b)=hg_{1}\otimes(a\leftarrow g_{2})b
% \]
% and unit $1\otimes1$. This algebra is called the \emph{right
% smash product} of $H$ and $A$. 
% \end{exercise}

%\section{Actions on coalgebras}

\begin{definition}
\label{def:module_coalgebra}
\index{Module!coalgebra}
Let $H$ be a Hopf algebra. A left \emph{$H$-module-coalgebra}
is a coalgebra $C$ with a left $H$-module structure such that 
\[
\sum (h\rightarrow c)_{1}\otimes(h\rightarrow c)_{2} =\sum (h_{1}\rightarrow c_{1})(h_{2}\rightarrow c_{2})
\quad\text{and}\quad 
\epsilon(h\rightarrow c)  =\epsilon(h)\epsilon(c)
\]
for all $h\in H$ and $c\in C$. 
\end{definition}

Similarly, a right $H$-module-coalgebra 
is a coalgebra $C$ with a right
$H$-module structure such that 
\[
\sum (c\leftarrow h)_{1}\otimes(c\leftarrow h)_{2} =\sum (c_{1}\leftarrow h_{1})(c_{2}\leftarrow h_{2})
\quad\text{and}\quad 
\epsilon(c\leftarrow h)  =\epsilon(h)\epsilon(c)
\]
for all $h\in H$ and $c\in C$.

\begin{exercise}
\index{Coadjoint!action}
Let $H$ be a finite-dimensional Hopf algebra and 
\[
(a\rightharpoonup f)(b)=f(ba)
\quad\text{and}\quad 
(f\leftharpoonup a)(b)=f(ab)
\]
for all $a,b\in H$ and $f\in H^{*}$. The \emph{left coadjoint action}
of $H$ on $H^{*}$ is 
\[
h\triangleright f=\sum (h_{1}\rightharpoonup f\leftharpoonup S^{-1}h_{2})=\sum f(S^{-1}h_{2}?h_1),
\]
where $f(?)$ means the function $x\mapsto f(x)$. Prove that
$(H^*)^\mathrm{cop}$ is a left $H$-module-coalgebra via the left coadjoint
action. 
\end{exercise}

Similarly,  a right $(H^*)^\mathrm{cop}$-module-coalgebra is given by the \emph{right coadjoint action} of $H$ on $H^{*}$, that is
\[
f\triangleleft h=\sum (S^{-1}h_{1}\rightharpoonup f\leftharpoonup h_{2})=\sum f(h_2?S^{-1}h_{1}).
\]

\begin{example}
Let $G$ be a finite group and $H=K[G]$. Let $\{e_g:g\in G\}$ be a basis of $H$. Then 
$y\rightharpoonup e_{x}=e_{xy^{-1}}$ (resp. $e_{x}\leftharpoonup
y=e_{y^{-1}x}$) defines a left (resp. right) $H$-module structure over $H^{*}$. In this case, 
the left coadjoint action of $H$ over $H^{*}$ is 
\[
y\triangleright e_{x}=y\rightharpoonup e_{x}\leftharpoonup y^{-1}=e_{xyx^{-1}}.
\]
\end{example}

\begin{exercise}
\index{Regular action}
Let $H$ be a Hopf algebra. Prove that 
Prove that $H$ is a left $H$-module-coalgebra with 
the left \emph{regular action} of $H$ on itself:
$h\rightarrow g=gh$ for all
$h,g\in H$. 
\end{exercise}

%\section{Coactions on algebras}
% Recall that a \emph{$H$-comodule} is a pair $(V,\delta)$,
% where $V$ is a vector space and $\delta:V\to H\otimes V$ is a linear
% map such that 
% \begin{align*}
% (\id\otimes\delta)\delta & =(\Delta\otimes\id)\delta,\\
% (\epsilon\otimes\id)\delta & =\id.
% \end{align*}
% We write $\delta(v)=v_{-1}\otimes v_{0}$. Similarly, a \emph{right
% $H$-comodule} is a pair $(V,\delta)$, where $\delta:V\to V\otimes H$
% is a linear map such that 
% \begin{align*}
% (\id\otimes\Delta)\delta & =(\delta\otimes\id)\delta,\\
% (\id\otimes\epsilon)\delta & =\id.
% \end{align*}
% In this case we write $\delta(v)=v_{0}\otimes v_{1}$.

\begin{definition}
\index{Comodule!algebra}
Let $H$ be a Hopf algebra. An algebra $A$ is said to be a left
\emph{$H$-comodule-algebra} if $A$ is a left $H$-comodule such that 
\begin{align*}
\sum (1_A)_{-1}\otimes (1_A)_0 =\sum 1_{H}\otimes1_{A},
\quad\text{and}\quad 
\sum (ab)_{-1}\otimes (ab)_0 =\sum a_{-1}b_{-1}\otimes a_{0}b_{0}
\end{align*}
for all $a,b\in A$. 
\end{definition}

%\section{Coactions on coalgebras}

\begin{definition}
\index{Comodule!coalgebra}
Let $H$ be a Hopf algebra. A coalgebra $C$ is said to be a left
\emph{$H$-comodule-coalgebra} if $C$ 
is a left $H$-comodule such that 
\begin{align*}
\sum c_{-1}\epsilon(c_{0}) & =\epsilon(c)1
\quad\text{and}\quad 
\sum (c_{1})_{-1}(c_{2})_{-1}\otimes(c_{1})_{0}\otimes(c_{2})_{0}  =\sum c_{-1}\otimes(c_{0})_{1}\otimes(c_{0})_{2}
\end{align*}
for all $c\in C$.
\end{definition}

\begin{exercise}
\index{Coadjoint!coaction}
Prove that a Hopf algebra 
$H$ is a left $H$-comodule-coalgebra via the left \emph{coadjoint coaction}
of $H$ on $H$, that is $\mathrm{coadj}(h)=\sum h_{1}S(h_{3})\otimes h_{2}$ for $h\in H$. 
\end{exercise}

\begin{exercise}
Let $H$ be a Hopf algebra, $C$ a coalgebra and $f\in\Hom(C,H)$
be a coalgebra map with convolution inverse $g$. Prove that $C$ 
is a left $H$-comodule coalgebra with
$\delta\colon H\to H\otimes C$, 
$\delta(c)=\sum f(c_{1})g(c_{3})\otimes c_{2}$. 
\end{exercise}

The following exercise introduces the 
\emph{left smash coproduct}.

\begin{exercise}
\label{xca:smash_coleft}
\index{Smash!coproduct}
Let $H$ be a Hopf algebra, and $(C,\delta)$ be a left $H$-comodule
coalgebra. Prove that $C\otimes H$ is a coalgebra with coproduct
\[
\Delta(c\otimes h)=\sum \left(c_{1}\otimes c_{2,-1}h_{1}\right)\otimes\left(c_{2,0}\otimes h_{2}\right),
\]
and counit $\epsilon(c\times h)=\epsilon(c)\epsilon(h)$ for
all $c\in C$ and $h\in H$.  Moreover, the maps $C\otimes H\to C$,
$c\otimes h\mapsto c\epsilon(h)$, and $C\otimes H\to H$, $c\otimes h\mapsto
\epsilon(c)h$, are coalgebra surjections.
\end{exercise}

Assume that $C$ is a right $H$-comodule coalgebra. The \emph{right
smash coproduct} is then defined by 
\[
\Delta(h\otimes c)=\sum h_{1}\otimes c_{1,0}\otimes h_{2}c_{1,1}\otimes c_{2}
\]
for all $h\in H$ and $c\in C$.




